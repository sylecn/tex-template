%% -*- mode: latex; tex-command: "xelatex"; TeX-engine: xetex -*-
%% this file should be saved as utf-8 and processed by xelatex.
\documentclass[a4paper,10pt,notitlepage,openany]{book}
\usepackage{amsmath}
\usepackage[BoldFont,SlantFont,CJKchecksingle]{xeCJK}
\punctstyle{quanjiao}
\setmainfont{Bitstream Vera Serif}
\setCJKmainfont{SimSun}
%% line spacing
\linespread{1.3}
\title{升龙用户手册}
\author{宋远乐 <songyuanle@yy.com>}
\date{\today}
\usepackage[xetex]{hyperref}
\hypersetup{colorlinks=true,linkcolor=blue}
\begin{document}
\maketitle
\tableofcontents
\chapter{chapter 1 测试章节1}
\section{my section}
When you need ``some quote'', you write it like that.
You can have ``single quotes `a' inside quotes''.

\begin{quote}
  Thou shell not pass!
\end{quote}
\begin{table}[htp]
  \label{table:host-load}
  \caption{hosts has load less than 0.1}
  \begin{tabular}{|c|c|c|}
    \hline
    host1 & host2 & host3 \\
    \hline
    0.1 & 0.2 & 0.3 \\
    \hline
  \end{tabular}
\end{table}

Mr.~Smith was happy.

\chapter{chapter 2 test chapter 2}

How does English paragraph indent? Let's see. I'm the first paragraph. I'm the first paragraph. I'm the first paragraph. I'm the first paragraph. I'm the first paragraph. I'm the first paragraph. I'm the first paragraph. I'm the first paragraph. I'm the first paragraph.

This is the second paragraph. This is the second paragraph. This is the second paragraph. This is the second paragraph. This is the second paragraph. This is the second paragraph. This is the second paragraph. This is the second paragraph. This is the second paragraph. This is the second paragraph. 

This         is the third paragraph. This         is the third paragraph. This         is the third paragraph. This         is the third paragraph. This         is the third paragraph. This         is the third paragraph. This         is the third paragraph. This         is the third paragraph. This         is the third paragraph. This         is the third paragraph. 


\chapter{chapter 3 测试章节3}\label{chap3}
测试一下中文支持。xelatex能够比较好的支持中文。测试一下中英文混排怎么样。

Openstack是Rackspace和NASA合作推出的开源云平台。IBM也插手了一把。Linux发行版那
边Redhat和Ubuntu都加入了Openstack。这个项目最终能走多远现在还很难说。

中英文混排OK。

据说 中文字符间的    空格可以自动忽略。    测试一下看看。

果然可以。不错。

\appendix
\chapter{appendix}
Here is appendix.

Here is url link \href{http://localhost/}{localhost}.

An example ref here.
see Table~\ref{table:host-load} on page~\pageref{table:host-load}.

link to chap \ref{chap3} Chinese tests.
\end{document}
